\documentclass{article}
\title{Mikroelektronika összefoglaló jegyzet}
\usepackage{geometry}
 \geometry{
 a4paper,
 total={170mm,257mm},
 left=20mm,
 top=20mm,
 }
\begin{document}
\maketitle

\section{Az integrált áramkörök gyártástechnológiája I. – Alapfogalmak}
\subsection{A szilícium technológia (Silicon manufacturing process)}

A mai integrált áramkörök túlnyomó többsége szilícium alapanyagú horodozón készül.
Egyéb célokra (pl.: optoelektronikai, nagyfrekvenciás, teljesítményelektronikai eszközök)
készülhetnek más, legtöbbször vegyületfélvezető anyagból (pl. szilícium-karbid, galliumarzenid stb.), melyek azonban inkább diszkrét vagy alacsony integráltságú eszközök.

A szilícium alapú integrált áramköri gyártástechnológia eleinte a szilícium jó elektromos
tulajdonságainak köszönhette a széles körű elterjedését, melyhez hozzájárult a könnyen
növeszthető, jó szigetelő és maszkoló rétegként egyaránt használható oxidja: a szilícium
dioxid (SiO2). Mindezek eredményeképpen mára a szilícium alapanyagra épülő
gyártástechnológia annyira fejletté vált, hogy a mai IC-k integráltsági szintjét és gyártási
költségét más félvezetőkkel egyelőre megközelíteni se lehet. Azért, hogy erről a területről
teljesebb képet kapjunk a következőkben a szilíciummal, mint félvezető alapanyaggal, és a
szilícium alapanyagú hordozóra planártechnológiával készült integrált áramkörök
gyártástechnológiájának néhány fontos aspektusával ismerkedhetünk meg.

Az integrált áramkörökben az áramkör összes tranzisztorát és egyéb elektronikus
alkatrészét (dióda, ellenállás, stb.) egy chipen, azaz egyetlen egy darab szilícium chip-ben
alakítják ki. Ehhez a kezdetben egységes szilíciumtömb anyagtulajdonságait lokálisan meg
kell változtatni, hiszen csak így alakulhatnak ki a különböző elektronikus alkatrészekhez (pl.:
tranzisztorok, diódák) szükséges, eltérő anyagtulajdonságokkal rendelkező rétegek. Integrált
áramkörökben nagyszámú eszközt alakítanak ki (modern processzorokban akár több
milliárd tranzisztor lehet egy chip-en), és az egyes eszközök elválasztása is eltérő
anyagparaméterű részekkel történik. A monolit (azaz egy alapanyagtömbben kialakított)
integrált áramkörök kialakításához tehát szükségünk van egy olyan technológiára, amelynek
segítségével a tömbi szilícium anyagi (főként elektromos) paramétereit csak és kizárólag a
kívánt helyen változtathatóak meg.

Az alapanyag villamos paramétereit általában adalékolással, azaz idegen atomok
szándékos, célzott és kontrolált bevitelével változtatjuk meg lokálisan, így kialakítva az aktív
eszközökhöz szükséges donor és akceptor adalékolású (n és p-típusú) régiók, melyek a
félvezető elektron és lyukvezetési képességeit határozzák meg. [1] Szilícium félvezető
esetén adalékanyagnak leggyakrabban foszfor (P) és bór (B) anyagot alkalmaznak. 

Az adalékolás mellet léteznek további technológia lépések, amelyeket csak lokálisan
szeretnénk végrehajtani, mint például a tömbi szilícium vagy egyéb rétegek lokális
eltávolítása, marása vagy éppen ennek az ellentéte, egyes rétegek lokális növesztése.
Ahhoz, hogy ezeket a lépéseket csak lokálisan végezhessük el (tehát mintázatot tudjunk
kialakítani), szükség van egy olyan ún. maszkoló rétegre, amely megvédi az alatta lévő alapanyagot azokon a helyeken, ahol az egyes folyamatokat nem szeretnénk végrehajtani.
Ennek a maszkoló rétegnek tehát nem a teljes felületen, hanem csak meghatározott
helyeken (lokálisan) kell a felületen lennie, ezért a maszkoló rétegeken is mintázatot kell
kialakítani. Ez általában úgy történik, hogy a maszkoló réteget a teljes felületre felviszik
(például SiO2 esetén növesztéssel), majd fotolitográfiával (2.6. fejezet) jelölik ki azokat a
helyeket, ahol nincs szükség maszkoló rétegre, és innen egy szelektív (például csak a
megvilágított maszk anyagot támadó) marószer segítségével eltávolítjuk a nem kívánt
részeket. A megfelelő mintázattal rendelkező maszkoló réteggel borított félvezetőn pedig
már elvégezhetjük az adott – mintázatnak megfelelően, csak lokálisan – elvégzendő
gyártástechnológiai lépéseket (adalékolás, maratás, rétegnövesztés stb.). A különböző
adalékolású részeket (2.7. fejezet), felületen futó vezetékeket, stb. tehát gyakorlatilag a kész
integrált áramkört az előzőekben vázolt folyamat (maszkolás, fotolitográfia, mintakialakítás)
gondosan megtervezett sorrendben történő ismétlésével alakíthatjuk ki. A szilícium alapú
integrált áramkörök gyártástechnológiája így különböző részfolyamatokból épül fel, ahol
minden egyes részfolyamat lényegében a maszkoló réteg felviteléből, maszkoló réteg
mintázásából és az alapanyag jellemzőinek változtatására szolgáló technológia lépésből áll. 

\subsection{A szilícium mechanikai és kristálytani tulajdonságai}

A szilícium a periódusos rendszer 14-es rendszámú eleme, kb. 1400°C fokig szilárd
halmazállapotú. Négy vegyértékű, így a kristályrácsában minden atomnak négy legközelebbi
szomszédja van, ezek a vegyérték elektronok segítségével kapcsolódnak össze. Elemi
cellája egy felületen középpontos, köbös szerkezet, amelybe egy hasonló szerkezet van
beleágyazva a testátló irányában 1/4 rácsállandóval eltolva. Ez a gyémántrács, amelynek a
szilícium a kiváló mechanikai tulajdonságait köszönheti. Az elemi cellát és a szilícium
legfontosabb kristálysíkjait a 2-1. ábra mutatja. A szilícium rácsálandója, azaz az elemi
cellájának (2-1. ábrán látható kocka) élhossza 0,543 nm, ami azt jelenti, hogy az egykristály
szilícium rácsában az egyes szilícium atommmagok egymástól kb. 5 Angström távolságban
helyezkednek el. 

KÉP

A rácsszerkezet a szilícium egyik legfontosabb tulajdonsága, mely meghatározza villamos
és mechanikai paramétereit egyaránt. Jó félvezető alapanyagként csakis a kevés hibával
rendelkező, egykristályos szilícium jöhet számításba. A szilícium egy fontos tulajdonsága,
hogy egyes kristálytani irányokban a gyártástechnológia során másképp viselkedik (például
különböző az adalékatamok behatolási mélysége), ezért nem mindegy, hogy milyen
orientációjú a szeletünk (azaz a szelet felületét melyik kristálytani sík alkotja). A három
leggyakrabban használt orientáció kristálysíkjait a 2-1. ábra mutatja be. Egyes lúgos
marószerek pl. lényegesen gyorsabban marják az <100> kristálysíkokat, mint az <111>
síkot, amelyet alakzatok marására szokás kihasználni. Bipoláris integrált áramkörökhöz
előszeretettel használnak <111>-es orientációjú szeleteket, mivel ezeken a szeleteken
könnyebben és gyorsabban növeszthető epitaxiális réteg. CMOS áramkörök gyártásához
azonban elterjedtebb az <100> orietációjú szeletek, mert ezeknél az ionimplantálás (2.7.2.
fejezet) végezhető el könnyebben. 

KÉP

Ha a szilícium mechanikai tulajdonságait tekintjük, akkor az 1. táblázatból láthatjuk, hogy
egy rendkívül kemény és nagy szakítószilárdságú anyagról van szó. Sűrűsége ezzel
ellentétben igen alacsony. Összehasonlításképpen láthatjuk, hogy annak ellenére, hogy
könnyebb, mint az alumínium, szakítószilárdsága meghaladja a titánét. Ezen tulajdonsága
miatt előszeretettel alkalmazzák mikromechanikai rendszerekben alapanyagként. Egyetlen
rossz mechanikai tulajdonsága a ridegsége, melynek következtében már igen csekély
alakváltozás esetén is törik, amelyet a mikrorendszerek tervezésekor figyelembe kell venni.
Ezzel szemben a megengedett tartományon belüli tartós és ismételt igénybevétel esetén
sincs szerkezeti, illetve szilárdsági változás, vagyis sem rugalmatlan alakváltozás, sem
pedig fáradás nem lép fel. További jó fizikai tulajdonságai közé tartozik a viszonylag magas
hővezetése, ami megkönnyíti a szilíciumból készült chipek hűtését. 



\end{document}